\chapter{linux配置}

\section{CMake安装和常用库的CMakeLists.txt的写法}
CMake是一个跨平台的生成工具,%
主要使用平台无关的CMakeLists.txt生成平台相关的Makefile文件。%

CMake的安装:\\
\noindent\textbf{\color{magenta}\$ tar -zvxf cmake.tar.gz}\\
\noindent\textbf{\color{magenta}\$ cd cmake}\\
\noindent\textbf{\color{magenta}\$ ./configure --prefix=to/install/path -\,-qt-gui}\\
\noindent\textbf{\color{magenta}\$ make -j4}\\
\noindent\textbf{\color{magenta}\$ make install}\\

\subsection{Boost}
\noindent\textbf{\color{magenta}Boost库的CMakeList.txt的写法:}\\
cmake\_minimum\_required(VERSION 3.0)\\

\noindent{}set(CMAKE\_PREFIX\_PATH "/opt/boost/install")\\
set(Boost\_USE\_STATIC\_LIBS OFF)\\
set(Boost\_USE\_MULTITHREADED ON)\\
set(Boost\_STATIC\_RUNTIME OFF)\\

\noindent{}find\_package(Boost)\\
include\_directories(\${Boost\_INCLUDE\_DIRS})\\

\noindent{}message(STATUS \${Boost\_INCLUDE\_DIRS})\\
message(STATUS \${Boost\_LIBRARIES})\\

\noindent{}add\_subdirectory(src)\\
其中Boost\_INCLUDE\_DIRS必须显示的添加到头文件的搜索路径中。

\subsection{OpenCV}
\noindent\textbf{\color{magenta}OpenCV库的CMakeLists.txt的写法:}

注意OpenCV的库的搜索结果是放在OpenCV\_LIBS这个变量中的,%
且头文件无需添加到搜索路径中。%

\subsection{ROOT}
\noindent\textbf{\color{magenta}ROOT库的CMakeLists.txt的写法:}

注意ROOT的库的搜索结果是放在ROOT\_LIBRARIES这个变量中的,%
且头文件无需添加到搜索路径中。%

\subsection{Qt5}
\noindent\textbf{\color{magenta}Qt5库的CMakeLists.txt的写法:}

注意对于含有Q\_OBJECT宏的头文件,%
需要使用qt5\_wrap\_cpp(MOC\_SRC \${MOC\_HEADER})进行编译,%
对于界面文件需要使用qt5\_wrap\_ui(UIC\_SRC \${UIC\_UI})进行编译,%
对于资源文件需要使用qt5\_add\_resources(QRC\_SRC \${QRC})进行编译,%
最后把编译好的源文件和\*.cpp一同作为目标文件的源文件,%
且头文件无需添加到搜索路径中。%
库的依赖方式为target\_link\_libraries(target Qt5::Core Qt5::Widgets Qt5::Gui Qt5::其它模块)。





\section{OpenCV的安装}
这里我们主要给出Ubuntu下OpenCV的安装例子。%
首先我们要先安装OpenCV所需要的依赖,%
打开terminal输入以下的命令:\\
\textbf{\color{magenta}\$ sudo apt-get -y install libopencv-dev 
   build-essential 
   cmake 
   libdc1394-22 
   libdc1394-22-dev 
   libjpeg-dev 
   libpng12-dev 
   libtiff-dev 
   libjasper-dev 
   libavcodec-dev 
   libavformat-dev 
   libswscale-dev 
   libxine-dev  
   libgstreamer0.10-dev 
   libgstreamer-plugins-base0.10-dev 
   libv41-dev 
   libtbb-dev 
   libqt4-dev 
   libmp3lame-dev 
   libopencore-amrnb-dev 
   libopencore-amrwb-dev 
   libtheora-dev 
   libvorbis-dev 
   libxvidcore-dev 
   x264 
   v41-utils}

 接着我们从官网或者github下载自己需要版本的源代码,%
 并且下载opencv\_contrib的测试模块。%
 分别解压缩两个源代码。%
 
 \noindent\textbf{\color{magenta}\$ cd opencv-版本} \\
 \noindent\textbf{\color{magenta}\$ mkdir build} \\
 \noindent\textbf{\color{magenta}\$ cd build} \\
 \noindent\textbf{\color{magenta}\$ cmake -D CMAKE\_BUILD\_TYPE=RELEASE\\
   -D CMAKE\_INSTALL\_PREFIX=/full/path/to/opencv/install/dir \\
   -D INSTALL\_C\_EXAMPLES=ON \\
   -D BUILD\_EXAMPLES=ON \\
   -D OPENCV\_EXTRA\_MODULES\_PATH=/full/path/to/opencv-contrib/modules\\
   ../} \\
 \noindent\textbf{\color{magenta}\$ make -j4}\\
 \noindent\textbf{\color{magenta}\$ sudo make install}
剩下只需要修改下PATH和LD\_LIBRARY\_PATH环境变量的值。

%%% Local Variables:
%%% mode: latex
%%% TeX-master: t
%%% End:
